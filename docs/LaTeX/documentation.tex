\documentclass{report}
\usepackage[english]{babel}
\usepackage[utf8]{inputenc}
\usepackage[T1]{fontenc}
\usepackage[hidelinks]{hyperref}
\usepackage[fixlanguage]{babelbib}
\usepackage{pgfplots}
\pgfplotsset{width=7cm,compat=1.18}
\usepackage{float}
\usepackage{subcaption}
\usepackage{booktabs}
\usepackage{amsmath}
\usepackage{array}
\usepackage{amssymb}
\usepackage{systeme}
\usepackage{tikz}
\usepackage{verbatim}
\usepackage{listings}
\usepackage{lipsum}
\usepackage{cite}

\begin{document}

\pagenumbering{gobble}

\begin{titlepage}
    \begin{figure}[!htb]
        \centering
        \includegraphics[keepaspectratio=true,scale=0.5]{cherubinFrontespizio.eps}
    \end{figure}
    
    \begin{center}
        \LARGE{UNIVERSITÀ DI PISA}
        \vspace{5mm}
        \\ \large{DEPARTMENT OF COMPUTER ENGINEERING}
    \end{center}
    
    \vspace{15mm}
    \begin{center}
        {\LARGE{\bf Edge Computing\\Project Documentation}}
    \end{center}
    \begin{center}
        {
        \makeatletter    
        \@date
        \makeatother    
        }
    \end{center}
    \vspace{30mm}
    
    \begin{minipage}[t]{0.47\textwidth}
        {\large{Team:}{\normalsize\vspace{3mm} \bf\\ \large{Cavedoni F.\\Monaci M.\\Pinna F.}}}
    \end{minipage}
    
    \vspace{15mm}
    \hrulefill
    \\\centering{\large{ACADEMIC YEAR 2024/2025}}
\end{titlepage}

\newpage
\tableofcontents
\newpage

\pagenumbering{arabic}
\setcounter{page}{1}

\chapter{Introduction}
A cellular network is composed of M base stations, placed within a 2D floorplan of size L $\times$ H
according to a regular grid. Each base station also has edge computing capabilities, i.e., it can receive
computing tasks from cellular network's users and serve them at a rate equal to S instructions per
seconds following a First Come First Served (FCFS) policy. Assume that all base stations are
interconnected between each other via mesh topology.
\section{Problem description}
We consider N users placed at random locations \texttt{(x, y)} within the same 2D floorplan, where coordinates
\texttt{x} and \texttt{y} are random variables to be defined later. Each user generates a new computing task request
every T seconds, and each request consists of I instructions to be executed. T and I are exponentially
distributed RVs. In particular, a user sends each new task request to its serving base station (i.e., the
closest one), which in turn can follow one of methods below:
\begin{itemize}
    \item [\textbf{a)}] serve the request locally
    \item [\textbf{b)}] forward the request to the less-loaded base station
\end{itemize}

\section{Objectives}
This project aims to fulfill the following:
\begin{itemize}
    \item Evaluate the time required to complete a computing task for various values of N comparing method a) against method b)
    \item Evaluate the following scenarios:
    
    \texttt{x} and \texttt{y} are \textit{uniform distributed} random variables in range \[x\in[0, width] \quad y\in[0, height]\]

    \texttt{x} and \texttt{y} are distributed as a \textit{lognormal distribution} with parameters defined in \autoref{factors}
    \item \texttt{TODO\dots}
\end{itemize}

\section{Performance indexes}
\begin{itemize}
    \item Response time of the system (when a packet leaves the system)
    \item Response time of each queue
    \item Packet loss
    \item \texttt{TODO\dots}
\end{itemize}

\chapter{Modeling}
Our system is described by the following scheme
\begin{figure}[H]
    \centering
    \includegraphics[width=\textwidth]{img/immagine.png}
    \caption{Scheme}
    \label{scheme}
\end{figure}

As shown in \autoref{scheme}, each base station is modeled as an \texttt{M/M/1/k} system, as rates $\lambda_i$ and $\mu_i$ are exponentially distributed random variables $\forall \ i \in [1, M]$.

\section{Assumptions}
We make some assumption in order to simplify the implementation of the system.

\begin{enumerate}
    \item Rerouting propagation delay is constant for each base station.
    \item Every job entering a queue is destined to be served and cannot exit the queue in any other way.
    \item Each queue length is \textit{finite} with \texttt{k} slots.
    \item \textit{(work in progress)} Width and height of the grid are such that it results in a square area.
\end{enumerate}

\section{Factors description}\label{factors}
\begin{itemize}
    \item \texttt{N}: number of users
    \item \texttt{k}: length of each queue
    \item $\lambda_i$: Interarrival rate for base station $i$
    \item $\mu$: Service rate for each base station
    \item $\mu_{log}$: Average of lognormal distribution
    \item $\sigma_{log}$: Standard deviation of lognormal distribution
\end{itemize}

\chapter{Implementation}
\section{Modules}
The following modules have been defined:
\begin{itemize}
    \item \textbf{EdgeNetwork}: compound module which represents the system and hosts the following simple modules
    \begin{itemize}
        \item \textbf{BaseStation}: simple module which receives, precesses and (in case of scenario b) forwards packets sent by users according to the specified distribution. 
        \item \textbf{User}: simple module which generates and sends packets (with length dependent by the specified distribution) to the nearest base station.
    \end{itemize}    
\end{itemize}

\section{Modules' behaviour}
\subsection{EdgeComputingNetwork}
\begin{itemize}
    \item Hosts base stations and users physically and memorizes their parameters in order to retreive them during the simulation via parent pointers.
\end{itemize}
\subsection{BaseStation}
\begin{itemize}
    \item Receives packets from users in form of \texttt{cMessage} objects and depending on the current scenario performs one of the following actions:
    \begin{itemize}
        \item [\textbf{Locally managed:}] if the queue has enough free slots the base station enqueues the packet to be processed, ignoring all other base stations.
        \item [\textbf{Forwarding:}] when a packet is received the base station searches for the base statio with the lowest load on their queue and forwards the packet to it. If and only if every other base station has a higher load, the packet is served locally.
    \end{itemize}
    \item It also records statistics every time a packet is \textit{dropped} or \textit{forwarded} (as number of packets) and about \textit{queue length} and \textit{response time} (in order to compute its average).
\end{itemize}

\subsection{User}
\begin{itemize}
    \item Generates packets with length and rate which are exponentially distributed and sends them to the \textit{nearest}\footnote{According to the euclidean distance}
    % non so se è superfluo specificarlo
    \texttt{BaseStation}.
\end{itemize}

\chapter{Verification}
Right away we describe the tests made to verify if our system tends to remain stable under extreme conditions or degenerate ones.

\section{Scenario 1 - no users}
As we expect in this degenerate case the system remains stable as there is nothing for it to precess or forward.

\section{Scenario 2 - very high number of users}
In this case we test our system against an enormous number of users and just a few base stations ($M=2 \ N=100$)

The system results to be robust to this kind of issue, maintaining a stable \textit{queue length} and a consistent \textit{response time}, as shown in \autoref{avg_ql_2} and \autoref{avg_rt_2}

\begin{figure}[H]
    \centering
    \includegraphics[width=0.8\textwidth]{img/avg_ql_2.png}
    \caption{}
    \label{avg_ql_2}
\end{figure}
\begin{figure}[H]
    \centering
    \includegraphics[width=0.8\textwidth]{img/avg_rt_2.png}
    \caption{}
    \label{avg_rt_2}
\end{figure}

\section{Scenario 3 - very small service rate}
Now we want to know how the system reacts when the service rate is a lot lower than the interarrival rate.

Throughout simulations we find that the queue length tends to be stable after about $50s$ (\autoref{avg_ql_3}). Instead the response time has a \textit{linear behaviour} over time (\autoref{avg_rt_3}).

\begin{figure}[H]
    \centering
    \includegraphics[width=0.8\textwidth]{img/avg_ql_3.png}
    \caption{}
    \label{avg_ql_3}
\end{figure}
\begin{figure}[H]
    \centering
    \includegraphics[width=0.8\textwidth]{img/avg_rt_3.png}
    \caption{}
    \label{avg_rt_3}
\end{figure}

\section{Consistency test}
In this test we verify that the system produces coherent results in both configurations and those results to be feasible.

We compare the results of simulations in both `\textit{Locally managed}' and `\textit{Forwarding}' configuration.

It is clear that in both cases we end up with similar means ($\Delta=1.22$), even though forwarding configuration is slightly better. What catches the eye is the difference between the variation of the two cases, once again favoring the forwarding.

\begin{figure}[H]
    \begin{subfigure}{0.60\textwidth}
        \centering
        \includegraphics[width=0.8\textwidth]{img/avg_rt_a.png}
        \caption{Locally managed}
    \end{subfigure}
    \begin{subfigure}{0.60\textwidth}
        \centering
        \includegraphics[width=0.8\textwidth]{img/avg_rt_b.png}
        \caption{Forwarding}
    \end{subfigure}
\end{figure}

\section{Continuity test}
This test ensures that the system responds without discontinuity while gradually varying the parameters.

We inspect the response time while incrementing two parameters: the number of users and the interarrival rate.

\subsection{Increment of users - ($10, \ 50, \ 100$)}\label{nusers}
As we might expect the response time becomes higher with the rising of the number of users but the mean always remains under $10s$. Once the system is underway, it remains stable with the majority of the oscillations belonging to the case $N=50$ (\autoref{rt_50})

\begin{figure}[H]
    \begin{subfigure}{0.55\textwidth}
        \centering
        \includegraphics[width=0.8\textwidth]{img/rt_10.png}
        \caption{$N=10$}
    \end{subfigure}
    \begin{subfigure}{0.55\textwidth}
        \centering
        \includegraphics[width=0.8\textwidth]{img/rt_50.png}
        \caption{$N=50$}
        \label{rt_50}
    \end{subfigure}
    \begin{subfigure}{0.55\textwidth}
        \centering
        \includegraphics[width=0.8\textwidth]{img/rt_100.png}
        \caption{$N=100$}
    \end{subfigure}
    \caption{}
\end{figure}

\section{Increment of requests' load - ($\lambda=0.1,\ 1,\ 5$)}
In this case the mean response time remains comparable for the first two values of lambda, the third one shows a mean value of almost an order of magnitude of difference and a higher variability.

\begin{figure}[H]
    \begin{subfigure}{0.55\textwidth}
        \centering
        \includegraphics[width=0.8\textwidth]{img/rt_lambda_01.png}
        \caption{$\lambda = 0.1$}
    \end{subfigure}
    \begin{subfigure}{0.55\textwidth}
        \centering
        \includegraphics[width=0.8\textwidth]{img/rt_lambda_1.png}
        \caption{$\lambda = 1$}
        \label{rt_50}
    \end{subfigure}
    \begin{subfigure}{0.55\textwidth}
        \centering
        \includegraphics[width=0.8\textwidth]{img/rt_lambda_5.png}
        \caption{$\lambda = 5$}
    \end{subfigure}
    \caption{}
\end{figure}

\section{Monotonicity test}
This test evaluate if the output of the system follows a monotonic relation with any of the input parameters.

\subsection{Number of users}
This case has already been already analyzed in \autoref{nusers}.

\subsection{Service rate}
With the variation of the service rate ($\mu = 1000, \ 5000, \ 10000$) the mean response time changes drastically over the cases.

In particular in the first one there aer tho orders of magnitude of difference with the others.

\begin{figure}[H]
    \begin{subfigure}{0.55\textwidth}
        \centering
        \includegraphics[width=0.8\textwidth]{img/rt_sr_1k.png}
        \caption{$\mu = 1000$}
    \end{subfigure}
    \begin{subfigure}{0.55\textwidth}
        \centering
        \includegraphics[width=0.8\textwidth]{img/placeholder.jpg}
        \caption{$\mu = 5000$}
    \end{subfigure}
    \begin{subfigure}{0.55\textwidth}
        \centering
        \includegraphics[width=0.8\textwidth]{img/placeholder.jpg}
        \caption{$\mu = 10000$}
    \end{subfigure}
    \caption{}
\end{figure}

\section{Verification against the theoretical model}
Now we compare the results of various simulations with those predicted by a theoretical model, which is described as follows.

\subsection{Theoretical analysis}
This is based mostly on \textit{queueing theory}, in particular on the $M|M|1$ model which describes our base stations, remembering the fact that they are characterized by a queue of finite length $k$.

Nowing that the service rate is equal to the product of the \textit{number of instructions processed per second} and the \textit{interarrival rate}
\[\mu = S\cdot \lambda\]

we compute the \textit{response time} for both configurations we simulated:

\textbf{Locally managed:}
\[t_r=\frac{1}{\mu-\lambda}\]

\textbf{Forwarding:}
\[t_r=D+\frac{1}{\mu-\lambda}\]

Where $D$ is a constant delay intrinsic in the system (i.e. network delay or computational delay)

In \autoref{th_a} and \autoref{th_b} we compare the results of theoretical and practical analysis in case of locally managed and forwarding.

\begin{figure}[H]
    \begin{subfigure}{0.55\textwidth}
        \centering
        \includegraphics[width=0.8\textwidth]{img/th_rt_a.png}
        \caption{Case a.}
        \label{th_a}
    \end{subfigure}
    \begin{subfigure}{0.55\textwidth}
        \centering
        \includegraphics[width=0.8\textwidth]{img/th_rt_b.png}
        \caption{Case b.}
        \label{th_b}
    \end{subfigure}
    \caption{}
\end{figure}

\chapter{Calibration}
We now discuss how factors and other parameters have been chosen throughout the developement.

\section{Factors calibration}
The aim of this section is to fix the intervals of the factors to correctly reproduce reality in our system.

\subsection{Number of base stations}
This number depends on where is the area located geographically and on it's dimentions.
In suburban areas, are commonly spaced 2/3 km apart and in dense urban areas, they may be as close as 400/800 m apart\footnote{Source: \href{https://en.wikipedia.org/wiki/Cell_site}{Wikipedia}}.

\subsection{Number of users per base station}
This number changes in the range $[10, 50]$ depending on how much an area is effectively crowded.\footnote{Reference: \href{https://ieeexplore.ieee.org/document/8305375}{Propagation channel Characterization for 28 and 73 GHz Millimeter-Wave 5G Frequency Band}}

\chapter{Simulation Experiments}
\section{The effect of forwarding packets on the system}
We test our system in both cases A and B to see the effects of forwarding packets wether the packets are generated acording to a uniform distribution or a lognormal distribution.

The tests also take into account 3 factors:
\begin{itemize}
    \item \textbf{N}: (100, 250, 500)
    \item \textbf{Interarrival rate}: ($\frac{1}{0.1}$, $\frac{1}{0.5}$)
    \item \textbf{Size rate}: ($10^3,\ 10^4$)
\end{itemize}

Considering the average case as a reference:
\begin{itemize}
    \item $N=250$
    \item $I=\frac{1}{0.5}$
    \item $S=10^3$
\end{itemize}

\subsection{Size rate variation - Method A}
\subsubsection*{Case $S=10^3$}

\begin{figure}[H]
    \centering
    \includegraphics[width=\textwidth]{img/plots/log_1e3_A/resptime.png}
    \caption{$E[R]$ for the lognormal distribution}
\end{figure}

\begin{figure}[H]
    \centering
    \includegraphics[width=\textwidth]{img/plots/uni_1e3_A/resptime.png}
    \caption{$E[R]$ for the uniform distribution}
\end{figure}

The difference between the two distribution is remarkable, as in the lognormal case we find a higher mean response time and queue length.

\begin{figure}[H]
    \centering
    \includegraphics[width=\textwidth]{img/plots/log_1e3_A/qlen.png}
    \caption{$E[qlen]$ for the lognormal distribution}
\end{figure}

\begin{figure}[H]
    \centering
    \includegraphics[width=\textwidth]{img/plots/uni_1e3_A/qlen.png}
    \caption{$E[qlen]$ for the uniform distribution}
\end{figure}

In the uniform distribution the values are more compact around the mean, unlike the lognormal case, where is evident that there are one or more base stations dealing with a heavier load.

\begin{figure}[H]
    \centering
    \includegraphics[width=\textwidth]{img/plots/log_1e3_A/dropped.png}
    \caption{Dropped packets distribution (lognormal)}
\end{figure}

\begin{figure}[H]
    \centering
    \includegraphics[width=\textwidth]{img/plots/uni_1e3_A/dropped.png}
    \caption{Dropped packets distribution (uniform)}
\end{figure}

Since base stations use the policy where packets are served locally, a higher queue length might mean a higher number of discarded packets. Even though, the system can still sustain the load in the uniform case as the number of discarded packets is lower than $10$.

Not the same for the lognormal case, where the total of discarded packets is over $2.5\times10^5$ units.

\vspace{10mm}

\subsubsection*{Case $S=10^4$}

\begin{figure}[H]
    \centering
    \includegraphics[width=\textwidth]{img/plots/log_1e4_A/resptime.png}
    \caption{$E[R]$ for the lognormal distribution}
\end{figure}

\begin{figure}[H]
    \centering
    \includegraphics[width=\textwidth]{img/plots/uni_1e4_A/resptime.png}
    \caption{$E[R]$ for the uniform distribution}
\end{figure}

\begin{figure}[H]
    \centering
    \includegraphics[width=\textwidth]{img/plots/log_1e4_A/qlen.png}
    \caption{$E[qlen]$ for the lognormal distribution}
\end{figure}

\begin{figure}[H]
    \centering
    \includegraphics[width=\textwidth]{img/plots/uni_1e4_A/qlen.png}
    \caption{$E[qlen]$ for the uniform distribution}
\end{figure}

This experiment shows how the system can better sustain the load in the lognormal case dealing with bigger packets. Unlike the previous experiment, this time we find similar behaviours.

\begin{figure}[H]
    \centering
    \includegraphics[width=\textwidth]{img/plots/log_1e4_A/dropped.png}
    \caption{Dropped packets distribution (lognormal)}
\end{figure}

\begin{figure}[H]
    \centering
    \includegraphics[width=\textwidth]{img/plots/uni_1e4_A/dropped.png}
    \caption{Dropped packets distribution (uniform)}
\end{figure}

Also the distribution of dropped packets is similar, presenting more concentrated values in the lognormal case.

\subsection{Size rate variation - Method B}
\subsubsection*{Case $S=10^3$}

\begin{figure}[H]
    \centering
    \includegraphics[width=\textwidth]{img/plots/log_1e3_B/resptime.png}
    \caption{$E[R]$ for the lognormal distribution}
\end{figure}

\begin{figure}[H]
    \centering
    \includegraphics[width=\textwidth]{img/plots/uni_1e3_B/resptime.png}
    \caption{$E[R]$ for the uniform distribution}
\end{figure}

As we can see the performance of the system improves with respect to the method A, but only when dealing with the lognormal distribution. The mean response time in the uniform case gets higher due to the forwarding directed to the same couple of base stations, this behaviour is responsible to raise the mean response time.

\begin{figure}[H]
    \centering
    \includegraphics[width=\textwidth]{img/plots/log_1e3_B/qlen.png}
    \caption{$E[qlen]$ for the lognormal distribution}
\end{figure}

\begin{figure}[H]
    \centering
    \includegraphics[width=\textwidth]{img/plots/uni_1e3_B/qlen.png}
    \caption{$E[qlen]$ for the uniform distribution}
\end{figure}

The same reasoning can be applied to the mean leangth of each queue, as the plots show the exact same behaviour.

\begin{figure}[H]
    \centering
    \includegraphics[width=\textwidth]{img/plots/log_1e3_B/forwarded.png}
    \caption{Forwarded packets distribution (lognormal)}
\end{figure}

\begin{figure}[H]
    \centering
    \includegraphics[width=\textwidth]{img/plots/uni_1e3_B/forwarded.png}
    \caption{Forwarded packets distribution (uniform)}
\end{figure}

It is curious that in the lognormal case packets tend to be forwarded to one particular base station unlike the uniform case, where are forwarded more evenly. The lognormal case presents a higher number of forwarded packets ($5.9\times10^6$ vs $4.3\times10^6$).

Dropped packets are not mentioned as there has been none throughout the experiment.
% LOG
% Total Dropped Packets: 0
% Total Forwarded Packets: 5959327
% UNI
% Total Dropped Packets: 0
% Total Forwarded Packets: 4390846

\subsubsection*{Case $S=10^4$}

\begin{figure}[H]
    \centering
    \includegraphics[width=\textwidth]{img/plots/log_1e4_B/resptime.png}
    \caption{$E[R]$ for the lognormal distribution}
\end{figure}

\begin{figure}[H]
    \centering
    \includegraphics[width=\textwidth]{img/plots/uni_1e4_B/resptime.png}
    \caption{$E[R]$ for the uniform distribution}
\end{figure}

In this case it is cear that the system is stressed out in both the distribution with almost equal mean response time. All the values of the base stations are concentrated around the global mean, which represents the fact that all of them respond at almost the same speed.

\begin{figure}[H]
    \centering
    \includegraphics[width=\textwidth]{img/plots/log_1e4_B/qlen.png}
    \caption{$E[qlen]$ for the lognormal distribution}
\end{figure}

% \begin{figure}[H]
%     \centering
%     \includegraphics[width=\textwidth]{img/plots/uni_1e4_B/qlen.png}
%     \caption{$E[qlen]$ for the uniform distribution}
% \end{figure}

The same reasoning can be applied to the mean leangth of each queue, as the plots show the exact same behaviour.

Since for both cases the plots are the same, it is shown the one relative to the lognormal distribution only.

\begin{figure}[H]
    \centering
    \includegraphics[width=\textwidth]{img/plots/log_1e4_B/forwarded.png}
    \caption{Forwarded packets distribution (lognormal)}
\end{figure}

\begin{figure}[H]
    \centering
    \includegraphics[width=\textwidth]{img/plots/uni_1e4_B/forwarded.png}
    \caption{Forwarded packets distribution (uniform)}
\end{figure}

It is shown the same behaviour as in the method A, the only difference is that in the uniform case packets are forwarded more evenly among all the base stations.

We report a total of $9.7\times10^6$ forwarded packets for the uniform and $11.4\times10^6$ for the lognormal.

\begin{figure}[H]
    \centering
    \includegraphics[width=\textwidth]{img/plots/log_1e4_B/dropped.png}
    \caption{Dropped packets distribution (lognormal)}
\end{figure}

\begin{figure}[H]
    \centering
    \includegraphics[width=\textwidth]{img/plots/uni_1e4_B/dropped.png}
    \caption{Dropped packets distribution (uniform)}
\end{figure}

As we can expect base stations that forward the majority of the packets are the ones that discard few of them and vice versa. We encounter this phenomenon for both the distributions.

We report the same number of discarded packets both for the lognormal and the uniform distribution.

% UNI
% Total Dropped Packets: 12284013
% Total Forwarded Packets: 9762424
% LOG
% Total Dropped Packets: 12285154
% Total Forwarded Packets: 11440655
\appendix

\end{document}